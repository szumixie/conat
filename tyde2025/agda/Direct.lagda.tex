\begin{code}[hide]
{-# OPTIONS --cubical --guardedness --no-import-sorts --no-postfix-projections --allow-unsolved-metas #-}

module Direct where

open import Cubical.Foundations.Prelude
open import Cubical.Data.Empty
  renaming (rec to exfalso)
open import Cubical.Data.Maybe using (Maybe; nothing; just; just-inj; ¬just≡nothing)
import Cubical.Data.Nat as ℕ
open ℕ using (ℕ)
open import Introduction using (ℕ∞; pred; zero; suc; pred-inj)
open import Problem using (_+_; +-match)
\end{code}
In this section, we show some examples of how to do corecursion without running
into the guardedness issue.

To avoid the guardedness issue in Section~\ref{sec:problem}, we define
multiplication from scratch instead of reusing addition.

First, we match to check whether both arguments are zero:
\begin{code}[hide]
infixl 7 _×_ _×'_
mutual
\end{code}
\begin{code}
  _×_ : ℕ∞ → ℕ∞ → ℕ∞
  pred (x × y) = ×-match (pred x) (pred y)

  ×-match : Maybe ℕ∞ → Maybe ℕ∞ → Maybe ℕ∞
  ×-match nothing    y'         = nothing
  ×-match (just x')  nothing    = nothing
  ×-match (just x')  (just y')  = just (x' ×' y')
\end{code}

For the non-zero case, we define a separate operation \AgdaFunction{\_\times'\_}
such that $x\times'y = (x+1)\times(y+1)-1$. The idea for the function is to basically
count $y+1$ steps $x+1$ times. This means that in the function we have to be somehow
able to reset the original number of steps we have to make in a cycle.
We achieve that by defining a helper function which keeps track of the original \AgdaBound{y}.

\begin{code}
  _×'_ : ℕ∞ → ℕ∞ → ℕ∞
  x ×' y = ×'-helper x y y

  ×'-helper : ℕ∞ → ℕ∞ → ℕ∞ → ℕ∞
  pred (×'-helper x y y₀) =
    ×'-helper-match x (pred x) (pred y) y₀

  ×'-helper-match :
    ℕ∞ → Maybe ℕ∞ → Maybe ℕ∞ → ℕ∞ →
    Maybe ℕ∞
  ×'-helper-match x x' (just y') y₀ =
    just (×'-helper x y' y₀)
  ×'-helper-match x (just x') nothing y₀ =
    just (×'-helper x' y₀ y₀)
  ×'-helper-match x nothing nothing y₀ =
    nothing
\end{code}

Here is an example just showing the internal states after every step
on how this funtcion calculates the predecessor of $4\times3$:

$(3,2) \to (3,1) \to (3,0) \to (2,2) \to \dots \to (0,1) \to (0,0)$

It took $4 \times 3 - 1 = 11$ steps to get to $(0,0)$

To prove commutativity of addition, we introduce an operation for adding an
inductive natural number to a conatural number:

\begin{code}
infixl 6 _+ₗ_
_+ₗ_ : ℕ → ℕ∞ → ℕ∞
ℕ.zero +ₗ x = x
ℕ.suc n +ₗ x = suc (n +ₗ x)
\end{code}



\begin{code}
+ₗ-suc : ∀ n x → n +ₗ suc x ≡ suc (n +ₗ x)
\end{code}
\begin{code}[hide]
+ₗ-suc ℕ.zero x = refl
+ₗ-suc (ℕ.suc n) x = cong suc (+ₗ-suc n x)
\end{code}

\begin{center}
  \begin{tikzpicture}[xscale=8, yscale=.5]
    \draw (0,1) -- (1,1);
    \draw (0,0) -- (1,0);

    \draw (0,0.75) -- (0,1.25);
    \draw ($(1/3,0.75)$) -- ($(1/3,1.25)$);
    \draw (1,0.75) -- (1,1.25);

    \draw (0,-0.25) -- (0,0.25);
    \draw ($(2/3,-0.25)$) -- ($(2/3,0.25)$);
    \draw (1,-0.25) -- (1,0.25);

    \draw [-{To[length=1mm]}] (0,1) -- ($(2/9,1)$);
    \draw [-{To[length=1mm]}] (0,0) -- ($(2/9,0)$);

    \draw [decorate, decoration={brace, amplitude=2mm}]
      ($(2/9,1.75)$) -- ($(1/3,1.75)$) node [midway, above=2mm] {$x$};
    \draw [decorate, decoration={brace, amplitude=2mm}]
      ($(1/3,1.75)$) -- (1,1.75) node [midway, above=2mm] {$n+y$};

    \draw [decorate, decoration={brace, mirror, amplitude=2mm}]
      (0,-0.75) -- ($(2/9,-0.75)$) node [midway, below=2mm] {$n$};
    \draw [decorate, decoration={brace, mirror, amplitude=2mm}]
      ($(2/9,-0.75)$) -- ($(2/3,-0.75)$) node [midway, below=2mm] {$y$};
    \draw [decorate, decoration={brace, mirror, amplitude=2mm}]
      ($(2/3,-0.75)$) -- (1,-0.75) node [midway, below=2mm] {$n+x$};
  \end{tikzpicture}
\end{center}

At this point we know $x$ is finite and equals to $n$:
\begin{center}
  \begin{tikzpicture}[xscale=8, yscale=.5]
    \draw (0,1) -- (1,1);
    \draw (0,0) -- (1,0);

    \draw (0,0.75) -- (0,1.25);
    \draw ($(1/3,0.75)$) -- ($(1/3,1.25)$);
    \draw (1,0.75) -- (1,1.25);

    \draw (0,-0.25) -- (0,0.25);
    \draw ($(2/3,-0.25)$) -- ($(2/3,0.25)$);
    \draw (1,-0.25) -- (1,0.25);

    \draw [-{To[length=1mm]}] (0,1) -- ($(4/9,1)$);
    \draw [-{To[length=1mm]}] (0,0) -- ($(4/9,0)$);

    \draw [decorate, decoration={brace, amplitude=2mm}]
      (0,1.75) -- ($(1/3,1.75)$) node [midway, above=2mm] {$n$};
    \draw [decorate, decoration={brace, amplitude=2mm}]
      ($(4/9,1.75)$) -- (1,1.75) node [midway, above=2mm] {$n+y$};

    \draw [decorate, decoration={brace, mirror, amplitude=2mm}]
      ($(4/9,-0.75)$) -- ($(2/3,-0.75)$) node [midway, below=2mm] {$y$};
    \draw [decorate, decoration={brace, mirror, amplitude=2mm}]
      ($(2/3,-0.75)$) -- (1,-0.75) node [midway, below=2mm] {$n$};
  \end{tikzpicture}
\end{center}

\begin{code}[hide]
mutual
\end{code}
\begin{code}
  +-comm : ∀ x y → x + y ≡ y + x
  +-comm x y = +-comm-helper₁ ℕ.zero x x y y refl refl

  +-comm-helper₁ :
    ∀ n x x₀ y y₀ → n +ₗ x ≡ x₀ → n +ₗ y ≡ y₀ → x + y₀ ≡ y + x₀
  pred (+-comm-helper₁ n x x₀ y y₀ nx ny i) =
    +-comm-helper₁-match n x (pred x) x₀ y (pred y) y₀ refl refl nx ny i

  +-comm-helper₁-match :
    ∀ n x x' x₀ y y' y₀ →
    pred x ≡ x' → pred y ≡ y' → n +ₗ x ≡ x₀ → n +ₗ y ≡ y₀ →
    +-match x' y₀ ≡ +-match y' x₀
  +-comm-helper₁-match n x nothing x₀ y nothing y₀ px py nx ny =
    cong pred
      (sym ny ∙ cong (n +ₗ_) (pred-inj py ∙ pred-inj {zero} (sym px)) ∙ nx)
  +-comm-helper₁-match n x nothing x₀ y (just y') y₀ px py nx ny =
    cong pred (sym ny ∙ cong (n +ₗ_) (pred-inj py) ∙ +ₗ-suc n y') ∙
    cong just
      ( +-comm-helper₂ n y' (n +ₗ y') refl ∙
        cong (y' +_) (cong (n +ₗ_) (pred-inj (sym px)) ∙ nx))
  +-comm-helper₁-match n x (just x') x₀ y nothing y₀ px py nx ny =
    cong just
      ( cong (x' +_) (sym ny ∙ cong (n +ₗ_) (pred-inj py)) ∙
        sym (+-comm-helper₂ n x' (n +ₗ x') refl)) ∙
    cong pred (sym (+ₗ-suc n x') ∙ cong (n +ₗ_) (pred-inj (sym px)) ∙ nx)
  +-comm-helper₁-match n x (just x') x₀ y (just y') y₀ px py nx ny =
    cong just
      (+-comm-helper₁ (ℕ.suc n) x' x₀ y' y₀
        (sym (+ₗ-suc n x') ∙ cong (n +ₗ_) (pred-inj (sym px)) ∙ nx)
        (sym (+ₗ-suc n y') ∙ cong (n +ₗ_) (pred-inj (sym py)) ∙ ny))
\end{code}
\begin{code}
  +-comm-helper₂ : ∀ n y y₀ → n +ₗ y ≡ y₀ → y₀ ≡ y + (n +ₗ zero)
  pred (+-comm-helper₂ n y y₀ ny i) =
    +-comm-helper₂-match n y (pred y) y₀ (pred y₀) refl refl ny i

  +-comm-helper₂-match :
    ∀ n y y' y₀ y₀' →
    pred y ≡ y' → pred y₀ ≡ y₀' → n +ₗ y ≡ y₀ →
    y₀' ≡ +-match y' (n +ₗ zero)
  +-comm-helper₂-match n y nothing y₀ y₀' py py₀ ny =
    sym py₀ ∙ cong pred (sym ny ∙ cong (n +ₗ_) (pred-inj py))
  +-comm-helper₂-match n y (just y') y₀ nothing py py₀ ny =
    exfalso
      (¬just≡nothing
        ( cong pred
            (sym (+ₗ-suc n y') ∙ cong (n +ₗ_) (pred-inj (sym py)) ∙ ny) ∙
          py₀))
  +-comm-helper₂-match n y (just y') y₀ (just y₀') py py₀ ny =
    cong just
      (+-comm-helper₂ n y' y₀'
        (just-inj _ _
          ( cong pred
              ( sym (+ₗ-suc n y') ∙
                cong (n +ₗ_) (pred-inj (sym py)) ∙
                ny) ∙
            py₀)))
\end{code}

extra equality arguments are to avoid transports, since they do not preserve
guardedness.

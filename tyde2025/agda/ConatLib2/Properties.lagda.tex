\begin{code}[hide]
{-# OPTIONS --safe --guardedness --cubical --no-postfix-projections #-}

module ConatLib2.Properties where

open import ConatLib2.Type
open import ConatLib2.Base
open import ConatLib2.Inspect
open import ConatLib2.Bisimilarity.Base

open import Agda.Primitive

open import Cubical.Foundations.Prelude
open import Cubical.Data.Maybe
open import Cubical.Data.Sum
open import Cubical.Data.Sigma
open import Cubical.Data.Empty
  renaming (rec to exfalso)
open import Cubical.Data.Unit
\end{code}

In this section we show the usage of the domain specific language and
equational reasoning to prove the commutativity property of addition
over conatural numbers. Using the definition of \AgdaFunction{\_+\_}
defined in the Introduction \textbf{HIVATKOZÁST BESZÚRNI RÁ?}, the following properties can be proven
directly without the language.

\begin{code}[hide]
unpred∘pred : ∀{n} → unpred (pred n) ≡ n
pred (unpred∘pred {n} i) = pred n

unpred∘just : ∀{n} → unpred (just n) ≡ suc n
pred (unpred∘just {n} i) = just n
\end{code}
\begin{code}
pred-inj : ∀{n k} → pred n ≡ pred k → n ≡ k
\end{code}
\begin{code}[hide]
pred (pred-inj e i) = e i
\end{code}
\begin{code}[hide]
idl+ : ∀ n → zero + n ≡ n
pred (idl+ n i) = pred n
\end{code}
\begin{code}
idr+ : ∀ n → n + zero ≡ n
\end{code}
\begin{code}[hide]
pred (idr+ n i) with pred n
... | nothing = nothing
... | just n' = just (idr+ n' i)
\end{code}
\begin{code}
sucl+ : ∀ n k → suc n + k ≡ suc (n + k)
\end{code}
\begin{code}[hide]
pred (sucl+ n k i) = just (n + k)
\end{code}

The addition is defined by pattern matching on the predecessor of the
first parameter, so we cannot easily handle the case where the \AgdaFunction{suc}
is on the second parameter. This is going to be the first property on which
we are going to have to use our language.

\begin{code}
`sucr+ : ∀ n k → n + suc k `≋ suc (n + k)
\end{code}
\begin{code}[hide]
`prove (`sucr+ n k) with pred n | inspect pred n
... | nothing | reveal eq1 = cong-just (
  k
    `≡⟨ idl+ k ⟨
  zero + k
    `≡⟨ cong (_+ k) (pred-inj eq1) ⟨
  n + k `∎
  )
... | just x  | reveal eq1 = cong-just (
  x + suc k
    `↺⟨ `sucr+ x k ⟩
  suc (x + k)
    `≡⟨ sucl+ x k ⟨
  suc x + k
    `≡⟨ cong (_+ k) (sym (pred-inj eq1)) ⟩
  n + k `∎
  )
\end{code}

For brevity we omit the proofs here. However, they are formalised and available in our repository. \textbf{??? LINK A REPÓRA?}

After proving \AgdaFunction{`sucr+}, we still need to obtain the proof over equality.
This is simply done by using the \AgdaFunction{run} function to turn the language's proof
into an equality.

\begin{code}
sucr+ : ∀ n k → n + suc k ≡ suc (n + k)
sucr+ n k = run (`sucr+ n k)
\end{code}
\begin{code}[hide]
assoc+ : ∀ a b c → (a + b) + c ≡ a + (b + c)
pred (assoc+ a b c i) with pred a
... | just x = just (assoc+ x b c i)
... | nothing = +-match (pred b) c
\end{code}

With everything set up, we are now able to prove commutativity. One of the advantages
of this method is that the proof is going to have nearly the same structure as the proof
for the same property in Introduction \textbf{HIVATKOZÁST BESZÚRNI RÁ?}. However, now the definition is going to
compile without any unsafe flags. We omit the uninteresting branches of the proof, namely when either
\AgdaBound{a} or \AgdaBound{b} is \AgdaFunction{zero}.

\begin{code}
`comm+ : ∀ a b → a + b `≋ b + a
`comm+-match : ∀ a a' b b' → a' ≡ pred a → b' ≡ pred b → Maybe~ _≋_ (+-match a' b) (+-match b' a)
\end{code}
\begin{code}[hide]
`comm+-match a nothing b nothing eq1 eq2 = subst2 (Maybe~ _≋_) eq2 eq1 nothing-refl
`comm+-match a nothing b (just b') eq1 eq2 = subst (λ x → Maybe~ _≋_ x (just (b' + a))) eq2 (cong-just (`path (sym (idr+ b')) `∙ `path (cong (b' +_) (pred-inj eq1))))
`comm+-match a (just a') b nothing eq1 eq2 = subst (Maybe~ _≋_ (just (a' + b))) eq1 (cong-just (`path (cong (a' +_) (pred-inj (sym eq2))) `∙ `path (idr+ a')))
\end{code}
\begin{code}
`comm+-match a (just a') b (just b') eq1 eq2 = cong-just (
  a' + b
    `≡⟨ cong (a' +_) (pred-inj (sym eq2)) ⟩
  a' + suc b'
    `↺⟨ `comm+ a' (suc b') ⟩
  suc b' + a'
    `≡⟨ sucl+ b' a' ⟩
  suc (b' + a')
    `≡⟨ sym (sucr+ b' a') ⟩
  b' + suc a'
    `≡⟨ cong (b' +_) (pred-inj eq1) ⟩
  b' + a `∎
  )
`prove (`comm+ a b) = `comm+-match a (pred a) b (pred b) refl refl
\end{code}

The only change in this proof with respect to the one in the Introduction is that
we need to switch up two steps. First, we need to use commutativity first to switch the
operands of \AgdaFunction{\_+\_} before making \AgdaFunction{suc} the outermost function.
This is because we do not have the \AgdaFunction{cong} property in our language.
However, this problem is easily solvable by just adding it to the language and
prove that it behaves the same way as the \AgdaFunction{cong} function in Cubical Agda itself.

Note that the proof only exists in our language so again we have to convert
it to make it use the equality using the \AgdaFunction{run} function the same
way as we did with the \AgdaFunction{sucr+} property.

\begin{code}
comm+ : ∀ a b → a + b ≡ b + a
comm+ a b = run (`comm+ a b)
\end{code}

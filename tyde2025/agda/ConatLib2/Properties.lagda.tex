\begin{code}[hide]
{-# OPTIONS --safe --guardedness --cubical --no-postfix-projections #-}

module ConatLib2.Properties where

open import ConatLib2.Type
open import ConatLib2.Base
open import ConatLib2.Inspect
open import ConatLib2.Bisimilarity.Base

open import Agda.Primitive

open import Cubical.Foundations.Prelude
open import Cubical.Data.Maybe
open import Cubical.Data.Sum
open import Cubical.Data.Sigma
open import Cubical.Data.Empty
  renaming (rec to exfalso)
open import Cubical.Data.Unit
\end{code}

With this we show the usage of the language and equational reasoning to prove
commutativity of addition over conatural numbers.
\begin{code}[hide]
unpred∘pred : ∀{n} → unpred (pred n) ≡ n
pred (unpred∘pred {n} i) = pred n

unpred∘just : ∀{n} → unpred (just n) ≡ suc n
pred (unpred∘just {n} i) = just n

pred-inj : ∀ {n k : ℕ∞} → pred n ≡ pred k → n ≡ k
pred (pred-inj e i) = e i

+-idₗ : ∀ n → zero + n ≡ n
pred (+-idₗ n i) = pred n

+-idᵣ : ∀ n → n + zero ≡ n
pred (+-idᵣ n i) with pred n
... | nothing = nothing
... | just n' = just (+-idᵣ n' i)

+-sucₗ : ∀ n k → suc n + k ≡ suc (n + k)
pred (+-sucₗ n k i) = just (n + k)
\end{code}

First, we prove that \AgdaFunction{suc} commutes with addition, in which we
already need to use transitivity, thus we prove it in the language. We omit the
proof here for brevity.
\begin{code}
`+-suc : ∀ x y → x + suc y `≈ suc (x + y)
\end{code}
\begin{code}[hide]
pred (`+-suc n k) with pred n | inspect pred n
... | nothing | reveal eq1 = cong-just (
  k
    `≡⟨ +-idₗ k ⟨
  zero + k
    `≡⟨ cong (_+ k) (pred-inj eq1) ⟨
  n + k `∎
  )
... | just x  | reveal eq1 = cong-just (
  x + suc k
    `↺⟨ `+-suc x k ⟩
  suc (x + k)
    `≡⟨ +-sucₗ x k ⟨
  suc x + k
    `≡⟨ cong (_+ k) (sym (pred-inj eq1)) ⟩
  n + k `∎
  )
\end{code}
We then interpret the proof in the language into an equality.
\begin{code}
+-suc : ∀ x y → x + suc y ≡ suc (x + y)
+-suc x y = interp (`step (`+-suc x y))
\end{code}
\begin{code}[hide]
assoc+ : ∀ a b c → (a + b) + c ≡ a + (b + c)
pred (assoc+ a b c i) with pred a
... | just x = just (assoc+ x b c i)
... | nothing = +-match (pred b) c
\end{code}

With everything set up, we are now able to prove commutativity. As in the
definition in Section~\ref{sec:dsl-mul}, the proof here has almost the same
structure as the one in Section~\ref{sec:problem}, but now it is guarded. The
slight difference is the reordering of the proof steps, which is to avoid the
need for congruence over \AgdaInductiveConstructor{suc}, though it is possible
to add this congruence as a constructor in the language as well. We omit the
non-recursive cases in \AgdaFunction{`+-comm-match}.
\begin{code}[hide]
mutual
\end{code}
\begin{code}
  `+-comm : ∀ x y → x + y `≈ y + x
  pred (`+-comm x y) =
    `+-comm-match x (pred x) y (pred y) refl refl

  `+-comm-match :
    ∀ x x' y y' → x' ≡ pred x → y' ≡ pred y →
    Maybe~ _≈_ (+-match x' y) (+-match y' x)
  `+-comm-match x (just x') y (just y') eq1 eq2 =
    cong-just
      ( x' + y         `≡⟨ cong (x' +_) (pred-inj (sym eq2)) ⟩
        x' + suc y'    `↺⟨ `+-comm x' (suc y') ⟩
        suc y' + x'    `≡⟨ pred-inj refl ⟩
        suc (y' + x')  `≡⟨ sym (+-suc y' x') ⟩
        y' + suc x'    `≡⟨ cong (y' +_) (pred-inj eq1) ⟩
        y' + x         `∎)
\end{code}
\begin{code}[hide]
  `+-comm-match a nothing b nothing eq1 eq2 = subst2 (Maybe~ _≈_) eq2 eq1 refl-nothing
  `+-comm-match a nothing b (just b') eq1 eq2 = subst (λ x → Maybe~ _≈_ x (just (b' + a))) eq2 (cong-just (`eq (sym (+-idᵣ b')) `∙ `eq (cong (b' +_) (pred-inj eq1))))
  `+-comm-match a (just a') b nothing eq1 eq2 = subst (Maybe~ _≈_ (just (a' + b))) eq1 (cong-just (`eq (cong (a' +_) (pred-inj (sym eq2))) `∙ `eq (+-idᵣ a')))
\end{code}
There is one difference in this proof compared to the one in
Section~\ref{sec:problem}, which is that we need to switch up two steps. First,
we need to use commutativity to switch the operands of \AgdaFunction{\_+\_}
before making \AgdaFunction{suc} the outermost function. This is because we do
not have congruence over \AgdaFunction{suc} in our language, but it could also
be added to the language.

The proof only exists in our language so again we convert it to an equality the
same way we did with the \AgdaFunction{+-suc} property.
\begin{code}
+-comm : ∀ x y → x + y ≡ y + x
+-comm x y = interp (`step (`+-comm x y))
\end{code}
This way, Agda sees that the equational reasoning is guarded, hence it accepts
the proof.

We have seen how we can define functions and prove equations using the method in
Danielsson~\cite{danielsson-beating}. However, this method can result in code
duplication, as can be seen in the definition of \AgdaFunction{predᴱ} and
\AgdaFunction{predᴱ-+-match} in Section~\ref{sec:dsl-mul}, where we had to
duplicate the definition of addition. If we want to use this definition of
multiplication and prove properties about it, we would need to separately prove
how the addition within the language is related to the addition on conatural
numbers.

A way to avoid the code duplication is to define addition using
\AgdaFunction{interp} and the \AgdaInductiveConstructor{\_`+\_} constructor.
However, if a coinductive proof depends on the congruence of an operation such
as addition, then we need to add the congruence of that operation to the proof
language, which means that when interpreting the language into equality, we will
have to explicitly prove the congruence of that operation. This proof of
congruence duplicates the structure of the definition of the operation. In the
next section, we merge operations and equality proofs into a single language to
avoid this problem.

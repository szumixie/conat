\begin{code}[hide]
{-# OPTIONS --cubical --guardedness --no-import-sorts --no-postfix-projections #-}

module Introduction where

open import Cubical.Foundations.Prelude
\end{code}
In dependent type theory, natural numbers are represented as an inductive type
with two constructors, one for zero and the other for the successor of a natural
number. In the Agda proof assistant, we write it as follows:
\begin{code}[hide]
module _ where private
\end{code}
\begin{code}
  data ℕ : Type where
    zero  :      ℕ
    suc   : ℕ →  ℕ
\end{code}
Categorically, natural numbers are the initial object in the category of
algebras over the $\mathbb{1} + {-}$ (or the ``Maybe'') endofunctor. So an
equivalent representation of natural numbers is an inductive type with a single
constructor containing a Maybe of a natural number:
\begin{code}[hide]
module _ where private
\end{code}
\begin{code}
  data Maybe (A : Type) : Type where
    nothing  :      Maybe A
    just     : A →  Maybe A

  data ℕ : Type where
    con : Maybe ℕ → ℕ
\end{code}

We can dualize natural numbers to get \emph{conatural numbers}, which are the
terminal object in the category of coalgebras over the $\mathbb{1} + {-}$
endofunctor. In Agda, it is a coinductive record type with one destructor into
Maybe of conatural numbers:
\begin{code}[hide]
open import Cubical.Data.Maybe using (Maybe; nothing; just)
\end{code}
\begin{code}
record ℕ∞ : Type where
  coinductive
  field
    pred : Maybe ℕ∞
\end{code}
\begin{code}[hide]
open ℕ∞ public
\end{code}
This destructor is the predecessor function which either fails or returns
another conatural number.

We can define elements of \AgdaRecord{ℕ∞} by copattern matching, that is, we
specify what the predecessor is for a particular element. As examples, zero does
not have a predecessor:
\begin{code}
zero : ℕ∞
pred zero = nothing
\end{code}
The predecessor of a successor of a number is just that number:
\begin{code}
suc : ℕ∞ → ℕ∞
pred (suc x) = just x
\end{code}
The predecessor of infinity is just infinity:
\begin{code}
∞ : ℕ∞
pred ∞ = just ∞
\end{code}
The above definition for \AgdaFunction{∞} is not structurally recursive, but it
is \emph{guarded}, that is, the recursive occurrence is after an instance of
copattern matching (\AgdaField{pred}), under only constructors
(\AgdaInductiveConstructor{just}). Agda uses guardedness to check whether a
corecursive defintion is productive. A definition is \emph{productive} when one
can compute the application of any finite amount of destructors on a corecursive
value in a finite amount of steps. Guardedness is sufficient to show
productivity but it is not necessary, thus Agda is too conservative about which
corecursive definitions it allows. (TODO: citations)

The conatural numbers can represent all natural numbers and an extra element for
infinity, however, it is not constructively isomorphic to $\mathbb{N} +
\mathbb{1}$, because all functions out of the conatural numbers must be
continuous. Computationally, finite amount of output can only depend on finite
amount of input. Topologically, \AgdaFunction{∞} is the limit of $0, 1, 2,
\dots$, which must be preserved. We can visualize the topological space as
follows:
\begin{center}
  \begin{tikzpicture}[
      scale=8,
      every node/.style={
        circle,
        fill,
        inner sep=0pt,
        minimum size=1mm,
        label position=south,
        label distance=1mm,
      }
    ]
    \path
    foreach \i in {0,...,7} {($({1-(2/3)^\i},0)$) node [label=\i] {}}
    ($({1-(2/3)^7/2},0)$) node [fill=none] {\dots}
    (1,0) node [label=∞] {};
  \end{tikzpicture}
\end{center}
As a result of this restriction, we cannot define a function that decides if an
element is equal to \AgdaFunction{∞}.

Our contribution in this paper is to prove that the conatural numbers form an
exponential commutative semiring by guarded corecursion, along the way showing
methods to keep the corecursion guarded. An \emph{exponential commutative
semiring} is a commutative semiring with a binary operation for exponentiation
satisfying the following equations:
\begin{align*}
  x^{yz} &= (x^y)^z & x^{y+z} &= x^y x^z & (xy)^z &= x^z y^z \\
  x^1 &= x & x^0 &= 1 & 1^x &= 1
\end{align*}

In Section~\ref{sec:problem}, we show how to define addition on conatural
numbers and prove that it is associative, at the same time, we show that if we
naïvely define multiplication and prove that addition is commutative, they get
rejected by Agda because they are not guarded. In the next sections, we show
three ways to avoid this issue, using multiplication and commutativity of
addition as running examples. In section~\ref{sec:direct}, we show how to

In section~\ref{sec:dsl}, we use Nils Anders Danielsson's
method~\cite{danielsson-beating}

In section~\ref{sec:quotiented}, we adapt the previous method to Cubical Agda,
making use of a mixed higher-inductive--coinductive type and univalence to
define and prove all the operations and equations of an exponential commutative
semiring at once.

\subsection{Formalization}

In vanilla Agda, it is not possible to prove non-trivial equations about
conatural numbers using the Martin-Löf identity type (TODO: cite), one needs to
either postulate the coinduction principle, or instead use a coinductively
defined equivalence relation, in which case one would need to prove that the
operations preserve that relation. As such, in this paper we use Cubical Agda,
where the coinduction principle and other equations can be directly proved
thanks to the interaction between copattern matching and the
interval~\cite{vezzosi-cubical}. As an example, we show the proof that the
predecessor function is injective:
\begin{code}
pred-inj : ∀ {x y} → pred x ≡ pred y → x ≡ y
pred (pred-inj p i) = p i
\end{code}

% we will not assume familiarity with cubical type theory
we will not use features of cubical type theory besides the interval from the
path/equality type

\subsection{Related work}

sized types, guarded recursion/later modality

decreasing boolean sequences

Naïm~\cite{favier-conat}

TypeTopology

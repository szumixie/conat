\begin{code}[hide]
{-# OPTIONS
  --safe
  --cubical
  --guardedness
  --no-import-sorts
  --hidden-argument-puns #-}

module Quotiented where

open import Cubical.Foundations.Prelude
open import Cubical.Foundations.Isomorphism using (Iso; iso; isoToPath)
open import Cubical.Foundations.Univalence using (ua→)
open import Cubical.Foundations.HLevels
open import Cubical.Reflection.RecordEquiv using (declareRecordIsoΣ)
import Cubical.Data.Empty as ⊥
open import Cubical.Data.Maybe
  using
    ( Maybe; nothing; just;
      isOfHLevelMaybe; just-inj; ¬nothing≡just; ¬just≡nothing)
  renaming (rec to matchMaybe)

module E where
\end{code}
Since we are working in Cubical Agda, which has higher inductive types, that is
datatypes with equality (path) constructors, we can adapt the approach in
Section~\ref{sec:dsl} to add all the algebraic operations and equations that we
are interested in into a single embedded language.

\subsection{Commutative semiring}

We will first show that the conatural numbers form a commutative semiring.

We mutually define expressions as a higher inductive type and head normal
expressions as a coinductive type. This is similar to the types in
Section~\ref{sec:dsl-mul}, except we add equations as constructors for
\AgdaDatatype{Expr}.
\begin{AgdaAlign}
\begin{code}
  data Expr : Type
  record NExpr : Type
\end{code}
Head normal expressions are represented as a coinductive record from which we
can extract the predecessor.
\begin{code}
  record NExpr where
    coinductive
    field pred : Maybe Expr
\end{code}
\begin{code}[hide]
  open NExpr public

  infixl 6 _+_
  infixl 7 _×_
\end{code}
In \AgdaDatatype{Expr}, we include all of the commutative semiring operations,
constants, and equations.
\begin{code}
  data Expr where
    _+_        : Expr → Expr → Expr
    +-assoc    : ∀ x y z → (x + y) + z ≡ x + (y + z)
    +-comm     : ∀ x y → x + y ≡ y + x

    zero       : Expr
    +-idₗ      : ∀ x → zero + x ≡ x

    _×_        : Expr → Expr → Expr
    ×-assoc    : ∀ x y z → (x × y) × z ≡ x × (y × z)
    ×-comm     : ∀ x y → x × y ≡ y × x

    one        : Expr
    ×-idₗ      : ∀ x → one × x ≡ x

    ×-distₗ-+  : ∀ x y z → (x + y) × z ≡ x × z + y × z
    ×-annihₗ   : ∀ x → zero × x ≡ zero
\end{code}
Since we are working in Cubical Agda, where equations/paths can have content, we
need to also say that the equations of \AgdaDatatype{Expr} are proof irrelevant,
that is, it is a set.
\begin{code}
    isSetExpr  : isSet Expr
\end{code}
We add a constructor to embed head normal expressions into expressions as in
Section~\ref{sec:dsl-mul}.
\begin{code}
    embed : NExpr → Expr
\end{code}
In addition, we add equations to relate taking the predecessor on head normal
expression. Note that here we depend on the destructor \AgdaField{pred} of
\AgdaRecord{NExpr}.
\begin{code}
    embed-zero  :
      ∀ x → pred x ≡ nothing → embed x ≡ zero
    embed-suc   :
      ∀ x x' → pred x ≡ just x' → embed x ≡ one + x'
\end{code}
Finally, a constructor for transitivity is added because the general
transitivity function does not preserve guardedness.
\begin{code}
    trans : ∀ {x y z} → x ≡ y → y ≡ z → x ≡ z
\end{code}
\end{AgdaAlign}
This is a mixed higher-inductive/coinductive definition.

We introduce an abbreviation for adding one to expressions, though it could also
be added as a constructor and an equation.
\begin{code}
  suc : Expr → Expr
  suc x = one + x
\end{code}
\begin{code}[hide]
  module ℕ∞Reasoning where
    infixl 6 _⟩+⟨_
    _⟩+⟨_ : ∀ {x x' y y'} → x ≡ x' → y ≡ y' → x + y ≡ x' + y'
    p ⟩+⟨ q = cong₂ _+_ p q

    +-idᵣ : ∀ x → x + zero ≡ x
    +-idᵣ x =
      x + zero ≡⟨ +-comm _ _ ⟩
      zero + x ≡⟨ +-idₗ _ ⟩
      x        ∎

    module _ {a x y z} (p : x + y ≡ z) where
      +-pullᵣ : a + x + y ≡ a + z
      +-pullᵣ =
        a + x + y   ≡⟨ +-assoc _ _ _ ⟩
        a + (x + y) ≡⟨ refl ⟩+⟨ p ⟩
        a + z       ∎

    module _ {a x y z} (p : x ≡ y + z) where
      +-pushᵣ : a + x ≡ a + y + z
      +-pushᵣ = sym (+-pullᵣ (sym p))

    module _ {a x} (p : zero ≡ x) where
      +-introₗ : a ≡ x + a
      +-introₗ =
        a        ≡⟨ sym (+-idₗ _) ⟩
        zero + a ≡⟨ p ⟩+⟨ refl ⟩
        x + a    ∎

      +-introᵣ : a ≡ a + x
      +-introᵣ =
        a        ≡⟨ sym (+-idᵣ _) ⟩
        a + zero ≡⟨ refl ⟩+⟨ p ⟩
        a + x    ∎

    module _ {a x} (p : x ≡ zero) where
      +-elimₗ : x + a ≡ a
      +-elimₗ = sym (+-introₗ (sym p))

      +-elimᵣ : a + x ≡ a
      +-elimᵣ = sym (+-introᵣ (sym p))

    module _ {a x y z w} (p : x + y ≡ z + w) where
      +-extendᵣ : a + x + y ≡ a + z + w
      +-extendᵣ =
        a + x + y   ≡⟨ +-assoc _ _ _ ⟩
        a + (x + y) ≡⟨ refl ⟩+⟨ p ⟩
        a + (z + w) ≡⟨ sym (+-assoc _ _ _) ⟩
        a + z + w   ∎

    +-sucₗ : ∀ x y → suc x + y ≡ suc (x + y)
    +-sucₗ x y = +-assoc _ _ _

    +-sucᵣ : ∀ x y → x + suc y ≡ suc (x + y)
    +-sucᵣ x y =
      x + suc y   ≡⟨ +-comm _ _ ⟩
      suc y + x   ≡⟨ +-sucₗ _ _ ⟩
      suc (y + x) ≡⟨ cong suc (+-comm _ _) ⟩
      suc (x + y) ∎

    infixl 7 _⟩×⟨_
    _⟩×⟨_ : ∀ {x x' y y'} → x ≡ x' → y ≡ y' → x × y ≡ x' × y'
    p ⟩×⟨ q = cong₂ _×_ p q

    ×-idᵣ : ∀ x → x × one ≡ x
    ×-idᵣ x =
      x × one ≡⟨ ×-comm _ _ ⟩
      one × x ≡⟨ ×-idₗ _ ⟩
      x       ∎

    module _ {a x y z} (p : x × y ≡ z) where
      ×-pullᵣ : a × x × y ≡ a × z
      ×-pullᵣ =
        a × x × y   ≡⟨ ×-assoc _ _ _ ⟩
        a × (x × y) ≡⟨ refl ⟩×⟨ p ⟩
        a × z       ∎

    module _ {a x y z} (p : x ≡ y × z) where
      ×-pushᵣ : a × x ≡ a × y × z
      ×-pushᵣ = sym (×-pullᵣ (sym p))

    module _ {a x} (p : one ≡ x) where
      ×-introᵣ : a ≡ a × x
      ×-introᵣ =
        a       ≡⟨ sym (×-idᵣ _) ⟩
        a × one ≡⟨ refl ⟩×⟨ p ⟩
        a × x   ∎

    module _ {a x} (p : x ≡ one) where
      ×-elimᵣ : a × x ≡ a
      ×-elimᵣ = sym (×-introᵣ (sym p))

    module _ {a x y z w} (p : x × y ≡ z × w) where
      ×-extendᵣ : a × x × y ≡ a × z × w
      ×-extendᵣ =
        a × x × y   ≡⟨ ×-assoc _ _ _ ⟩
        a × (x × y) ≡⟨ refl ⟩×⟨ p ⟩
        a × (z × w) ≡⟨ sym (×-assoc _ _ _) ⟩
        a × z × w   ∎

    ×-annihᵣ : ∀ x → x × zero ≡ zero
    ×-annihᵣ x =
      x × zero ≡⟨ ×-comm _ _ ⟩
      zero × x ≡⟨ ×-annihₗ _ ⟩
      zero     ∎

    ×-sucₗ : ∀ x y → suc x × y ≡ y + x × y
    ×-sucₗ x y =
      (one + x) × y   ≡⟨ ×-distₗ-+ _ _ _ ⟩
      one × y + x × y ≡⟨ ×-idₗ _ ⟩+⟨ refl ⟩
      y + x × y       ∎

    ×-sucᵣ : ∀ x y → x × suc y ≡ x + x × y
    ×-sucᵣ x y =
      x × suc y ≡⟨ ×-comm _ _ ⟩
      suc y × x ≡⟨ ×-sucₗ _ _ ⟩
      x + y × x ≡⟨ refl ⟩+⟨ ×-comm _ _ ⟩
      x + x × y ∎

    one-suc : one ≡ suc zero
    one-suc = sym (+-idᵣ _)

    unpred : Maybe Expr → Expr
    unpred nothing = zero
    unpred (just x') = suc x'

  open ℕ∞Reasoning
\end{code}

For convenience, we define an inductive predicate $\AgdaDatatype{IsPred}\ x'\ x$
which represents the $\AgdaField{pred}\ x \mathrel{\AgdaFunction{≡}} x'$
equation, but it can be pattern matched on so it simplifies the proofs.
\begin{code}
  data IsPred : Maybe Expr → Expr → Type where
    nothing  :         IsPred nothing    zero
    just     : ∀ x' →  IsPred (just x')  (suc x')
\end{code}
\begin{code}[hide]
  isPropIsPred-zero :
    ∀ {x' x} (p : nothing ≡ x') (q : zero ≡ x) r →
    PathP (λ i → IsPred (p i) (q i)) nothing r
  isPropIsPred-zero p q nothing =
    subst2 (λ e₁ e₂ → PathP (λ i → IsPred (e₁ i) (e₂ i)) nothing nothing)
      (isOfHLevelMaybe 0 isSetExpr _ _ refl p)
      (isSetExpr _ _ refl q)
      refl
  isPropIsPred-zero p q (just x') = ⊥.rec (¬nothing≡just p)

  isPropIsPred-suc :
    ∀ {x' x} x'′ (p : just x'′ ≡ x') (q : suc x'′ ≡ x) r →
    PathP (λ i → IsPred (p i) (q i)) (just x'′) r
  isPropIsPred-suc x'′ p q nothing = ⊥.rec (¬just≡nothing p)
  isPropIsPred-suc x'′ p q (just x') =
    subst2 (λ e₁ e₂ → PathP (λ i → IsPred (e₁ i) (e₂ i)) (just x'′) (just x'))
      (isOfHLevelMaybe 0 isSetExpr _ _ (cong just (just-inj _ _ p)) p)
      (isSetExpr _ _ (cong suc (just-inj _ _ p)) q)
      (cong IsPred.just (just-inj _ _ p))

  isPropIsPred : ∀ x' x → isProp (IsPred x' x)
  isPropIsPred _ _ nothing q = isPropIsPred-zero refl refl q
  isPropIsPred _ _ (just x') q = isPropIsPred-suc x' refl refl q
\end{code}

We want to define the predecessor function on expressions by induction. We
define the motive and methods (terminology from McBride~\cite{mcbride-view}), so
the termination checker does not get in the way. If we do recursion by pattern
matching over a higher inductive type, some implicit arguments get solved to an
expression that is not reduced and the termination checker complains. For the
motive, we want the predecessor of an expression and a proof that it is indeed
the predecessor in terms of the semiring operations.
\begin{code}
  record Pred (x : Expr) : Type where
    field
      pred    : Maybe Expr
      isPred  : IsPred pred x
\end{code}
\begin{code}[hide]
  open Pred
  unquoteDecl PredIsoΣ = declareRecordIsoΣ PredIsoΣ (quote Pred)

  Pred-path :
    ∀ {x y} {p : x ≡ y} {xᴾ yᴾ} →
    xᴾ .pred ≡ yᴾ .pred → PathP (λ i → Pred (p i)) xᴾ yᴾ
  Pred-path q i .pred = q i
  Pred-path {p} {xᴾ} {yᴾ} q i .isPred =
    isProp→PathP (λ i → isPropIsPred (q i) (p i)) (xᴾ .isPred) (yᴾ .isPred) i
\end{code}

We need \AgdaRecord{Pred} to be a set to be able to eliminate into it.
Fortunately, this fact is easily provable using the combinators provided by the
Agda Cubical library~\cite{agda-cubical}.
\begin{code}
  isSetPred : ∀ x → isSet (Pred x)
\end{code}
\begin{code}[hide]
  isSetPred x =
    isOfHLevelRetractFromIso 2 PredIsoΣ
      (isSetΣ (isOfHLevelMaybe 0 isSetExpr) λ x' →
        isProp→isSet (isPropIsPred _ _))
\end{code}

We can define the methods for each constructor in \AgdaDatatype{Expr}. The one
below is the method for addition:
\begin{code}[hide]
  mutual
\end{code}
\begin{code}
    infixl 6 _+ᴾ_
    _+ᴾ_ : ∀ {x y} → Pred x → Pred y → Pred (x + y)
    pred    (_+ᴾ_ {y}  xᴾ yᴾ) = +-pred (pred xᴾ) y (pred yᴾ)
    isPred  (_+ᴾ_      xᴾ yᴾ) = +-isPred (isPred xᴾ) (isPred yᴾ)
\end{code}
If we know the predecessor of $x$ and $y$, then we can define the predecessor of
$x + y$.
\begin{code}
    +-pred :
      Maybe Expr → Expr → Maybe Expr → Maybe Expr
    +-pred nothing    y y' = y'
    +-pred (just x')  y y' = just (x' + y)
\end{code}
We then need to prove that it is actually the predecessor, which we do by using
the equation constructors.
\begin{code}
    +-isPred :
      ∀ {x x' y y'} → IsPred x' x → IsPred y' y →
      IsPred (+-pred x' y y') (x + y)
    +-isPred {y} nothing    p = subst (IsPred _) (sym (+-idₗ _)) p
    +-isPred {y} (just x')  p =
      subst (IsPred _) (sym (+-assoc _ _ _)) (just (x' + y))
\end{code}

We can do the same for multiplication as well, here we show how to define the
predecessor. The structure is again the same as the one in
Section~\ref{sec:problem} as we can reuse the existing constructors.
\begin{code}
  ×-pred :
    Maybe Expr → Expr → Maybe Expr → Maybe Expr
  ×-pred nothing    y y'         = nothing
  ×-pred (just x')  y nothing    = nothing
  ×-pred (just x')  y (just y')  = just (y' + x' × y)
\end{code}
\begin{code}[hide]
  ×-isPred :
    ∀ {x x' y y'} →
    IsPred x' x → IsPred y' y → IsPred (×-pred x' y y') (x × y)
  ×-isPred nothing p = subst (IsPred _) (sym (×-annihₗ _)) nothing
  ×-isPred (just x') nothing = subst (IsPred _) (sym (×-annihᵣ _)) nothing
  ×-isPred {y} (just x') (just y') =
    subst (IsPred _)
      ( suc (y' + x' × y)  ≡⟨ sym (+-sucₗ _ _) ⟩
        suc y' + x' × y    ≡⟨⟩
        y + x' × y         ≡⟨ sym (×-sucₗ _ _) ⟩
        suc x' × y         ∎)
      (just (y' + x' × y))

  infixl 7 _×ᴾ_
  _×ᴾ_ : ∀ {x y} → Pred x → Pred y → Pred (x × y)
  _×ᴾ_ {y} xᴾ yᴾ .pred = ×-pred (xᴾ .pred) y (yᴾ .pred)
  (xᴾ ×ᴾ yᴾ) .isPred = ×-isPred (xᴾ .isPred) (yᴾ .isPred)

  +-assoc-pred :
    ∀ x' y y' z z' →
    +-pred (+-pred x' y y') z z' ≡ +-pred x' (y + z) (+-pred y' z z')
  +-assoc-pred nothing y y' z z' = refl
  +-assoc-pred (just x') y y' z z' = congS just (+-assoc _ _ _)

  +-assocᴾ :
    ∀ {x y z} xᴾ yᴾ zᴾ →
    PathP (λ i → Pred (+-assoc x y z i)) ((xᴾ +ᴾ yᴾ) +ᴾ zᴾ) (xᴾ +ᴾ (yᴾ +ᴾ zᴾ))
  +-assocᴾ {y} {z} xᴾ yᴾ zᴾ =
    Pred-path (+-assoc-pred (xᴾ .pred) y (yᴾ .pred) z (zᴾ .pred))
\end{code}

Below we show the method for the commutativity of addition, which has the same
structure as the proof in Section~\ref{sec:problem}, and it has less steps due
to pattern matching on \AgdaDatatype{IsPred}. Note that we do not need to define
the \AgdaField{isPred} component because \AgdaRecord{IsPred} is a proposition.
We also omit the less interesting parts.
\begin{code}
  +-comm-pred :
    ∀ {x x' y y'} → IsPred x' x → IsPred y' y →
    +-pred x' y y' ≡ +-pred y' x x'
  +-comm-pred (just x') (just y') =
    congS just
      (  x' + suc y'    ≡⟨ +-sucᵣ _ _ ⟩
         suc (x' + y')  ≡⟨ cong suc (+-comm _ _) ⟩
         suc (y' + x')  ≡⟨ sym (+-sucᵣ _ _) ⟩
         y' + suc x'    ∎)
\end{code}
\begin{code}[hide]
  +-comm-pred nothing nothing = refl
  +-comm-pred nothing (just y') =
    congS just (sym (+-idᵣ _))
  +-comm-pred (just x') nothing =
    congS just (+-idᵣ _)

  +-commᴾ :
    ∀ {x y} xᴾ yᴾ → PathP (λ i → Pred (+-comm x y i)) (xᴾ +ᴾ yᴾ) (yᴾ +ᴾ xᴾ)
  +-commᴾ xᴾ yᴾ = Pred-path (+-comm-pred (xᴾ .isPred) (yᴾ .isPred))

  zeroᴾ : Pred zero
  zeroᴾ .pred = nothing
  zeroᴾ .isPred = nothing

  +-idₗᴾ : ∀ {x} xᴾ → PathP (λ i → Pred (+-idₗ x i)) (zeroᴾ +ᴾ xᴾ) xᴾ
  +-idₗᴾ xᴾ = Pred-path refl

  ×-assoc-pred :
    ∀ x' y y' z z' →
    ×-pred (×-pred x' y y') z z' ≡ ×-pred x' (y × z) (×-pred y' z z')
  ×-assoc-pred nothing y y' z z' = refl
  ×-assoc-pred (just x') y nothing z z' = refl
  ×-assoc-pred (just x') y (just y') z nothing = refl
  ×-assoc-pred (just x') y (just y') z (just z') =
    congS just
      ( z' + (y' + x' × y) × z     ≡⟨ +-pushᵣ (×-distₗ-+ _ _ _) ⟩
        z' + y' × z + x' × y × z   ≡⟨ refl ⟩+⟨ ×-assoc _ _ _ ⟩
        z' + y' × z + x' × (y × z) ∎)

  ×-assocᴾ : ∀ {x y z} xᴾ yᴾ zᴾ →
    PathP (λ i → Pred (×-assoc x y z i)) ((xᴾ ×ᴾ yᴾ) ×ᴾ zᴾ) (xᴾ ×ᴾ (yᴾ ×ᴾ zᴾ))
  ×-assocᴾ {x} {y} {z} xᴾ yᴾ zᴾ =
    Pred-path (×-assoc-pred (xᴾ .pred) y (yᴾ .pred) z (zᴾ .pred))

  ×-comm-pred :
    ∀ {x x' y y'} → IsPred x' x → IsPred y' y →
    ×-pred x' y y' ≡ ×-pred y' x x'
  ×-comm-pred nothing nothing = refl
  ×-comm-pred nothing (just y') = refl
  ×-comm-pred (just x') nothing = refl
  ×-comm-pred (just x') (just y') =
    congS just
      ( y' + x' × suc y'  ≡⟨ +-pushᵣ (×-sucᵣ _ _) ⟩
        y' + x' + x' × y' ≡⟨ +-comm _ _ ⟩+⟨ ×-comm _ _ ⟩
        x' + y' + y' × x' ≡⟨ +-pullᵣ (sym (×-sucᵣ _ _)) ⟩
        x' + y' × suc x' ∎)

  ×-commᴾ :
    ∀ {x y} xᴾ yᴾ → PathP (λ i → Pred (×-comm x y i)) (xᴾ ×ᴾ yᴾ) (yᴾ ×ᴾ xᴾ)
  ×-commᴾ xᴾ yᴾ = Pred-path (×-comm-pred (xᴾ .isPred) (yᴾ .isPred))

  oneᴾ : Pred one
  oneᴾ .pred = just zero
  oneᴾ .isPred = subst (IsPred _) (sym one-suc) (just zero)

  ×-idₗ-pred : ∀ x x' → ×-pred (just zero) x x' ≡ x'
  ×-idₗ-pred x nothing = refl
  ×-idₗ-pred x (just x') = congS just (+-elimᵣ (×-annihₗ _))

  ×-idₗᴾ : ∀ {x} xᴾ → PathP (λ i → Pred (×-idₗ x i)) (oneᴾ ×ᴾ xᴾ) xᴾ
  ×-idₗᴾ {x} xᴾ = Pred-path (×-idₗ-pred x (xᴾ .pred))

  ×-distₗ-+-pred :
    ∀ x' y y' z z' →
    ×-pred (+-pred x' y y') z z' ≡
    +-pred (×-pred x' z z') (y × z) (×-pred y' z z')
  ×-distₗ-+-pred nothing y y' z z' = refl
  ×-distₗ-+-pred (just x) y y' z (just z') =
    congS just (+-pushᵣ (×-distₗ-+ _ _ _))
  ×-distₗ-+-pred (just x) y nothing z nothing = refl
  ×-distₗ-+-pred (just x) y (just y') z nothing = refl

  ×-distₗ-+ᴾ :
    ∀ {x y z} xᴾ yᴾ zᴾ →
    PathP (λ i → Pred (×-distₗ-+ x y z i))
      ((xᴾ +ᴾ yᴾ) ×ᴾ zᴾ) (xᴾ ×ᴾ zᴾ +ᴾ yᴾ ×ᴾ zᴾ)
  ×-distₗ-+ᴾ {y} {z} xᴾ yᴾ zᴾ =
    Pred-path (×-distₗ-+-pred (xᴾ .pred) y (yᴾ .pred) z (zᴾ .pred))

  ×-annihₗᴾ : ∀ {x} xᴾ → PathP (λ i → Pred (×-annihₗ x i)) (zeroᴾ ×ᴾ xᴾ) zeroᴾ
  ×-annihₗᴾ xᴾ = Pred-path refl

  sucᴾ : ∀ {x} → Pred x → Pred (suc x)
  sucᴾ = oneᴾ +ᴾ_

  transᴾ :
    ∀ {x y z} {p : x ≡ y} {q : y ≡ z} {xᴾ yᴾ zᴾ} →
    PathP (λ i → Pred (p i)) xᴾ yᴾ → PathP (λ i → Pred (q i)) yᴾ zᴾ →
    PathP (λ i → Pred (trans p q i)) xᴾ zᴾ
  transᴾ pᴾ qᴾ = Pred-path ((λ i → pᴾ i .pred) ∙ (λ i → qᴾ i .pred))

  embed-isPred : ∀ x x' → x .pred ≡ x' → IsPred x' (embed x)
  embed-isPred x nothing p = subst (IsPred _) (sym (embed-zero x p)) nothing
  embed-isPred x (just x') p =
    subst (IsPred _) (sym (embed-suc x x' p)) (just x')

  embedᴾ : ∀ x → Pred (embed x)
  embedᴾ x .pred = x .pred
  embedᴾ x .isPred = embed-isPred x (x .pred) refl

  embed-zeroᴾ : ∀ x p → PathP (λ i → Pred (embed-zero x p i)) (embedᴾ x) zeroᴾ
  embed-zeroᴾ x p = Pred-path p

  embed-sucᴾ :
    ∀ {x'} x x'ᴾ p →
    PathP (λ i → Pred (embed-suc x x' p i)) (embedᴾ x) (sucᴾ x'ᴾ)
  embed-sucᴾ {x'} x x'ᴾ p = Pred-path
    ( x .pred          ≡⟨ p ⟩
      just x'          ≡⟨ congS just (sym (+-idₗ _)) ⟩
      just (zero + x') ∎)
\end{code}

We can define the rest of the methods in a similar manner as the ones above.
Using these, we can recursively eliminate from expressions into
\AgdaRecord{Pred}.
\begin{code}
  ⟦_⟧ᴾ : (x : Expr) → Pred x
\end{code}
\begin{code}[hide]
  ⟦ isSetExpr x y p q i j ⟧ᴾ =
    isOfHLevel→isOfHLevelDep 2 (λ x → isSetPred x)
      ⟦ x ⟧ᴾ ⟦ y ⟧ᴾ (cong ⟦_⟧ᴾ p) (cong ⟦_⟧ᴾ q) (isSetExpr _ _ _ _) i j

  ⟦ x + y ⟧ᴾ = ⟦ x ⟧ᴾ +ᴾ ⟦ y ⟧ᴾ
  ⟦ +-assoc x y z i ⟧ᴾ = +-assocᴾ ⟦ x ⟧ᴾ ⟦ y ⟧ᴾ ⟦ z ⟧ᴾ i
  ⟦ +-comm x y i ⟧ᴾ = +-commᴾ ⟦ x ⟧ᴾ ⟦ y ⟧ᴾ i
  ⟦ zero ⟧ᴾ = zeroᴾ
  ⟦ +-idₗ x i ⟧ᴾ = +-idₗᴾ ⟦ x ⟧ᴾ i

  ⟦ x × y ⟧ᴾ = ⟦ x ⟧ᴾ ×ᴾ ⟦ y ⟧ᴾ
  ⟦ ×-assoc x y z i ⟧ᴾ = ×-assocᴾ ⟦ x ⟧ᴾ ⟦ y ⟧ᴾ ⟦ z ⟧ᴾ i
  ⟦ ×-comm x y i ⟧ᴾ = ×-commᴾ ⟦ x ⟧ᴾ ⟦ y ⟧ᴾ i
  ⟦ one ⟧ᴾ = oneᴾ
  ⟦ ×-idₗ x i ⟧ᴾ = ×-idₗᴾ ⟦ x ⟧ᴾ i
  ⟦ ×-distₗ-+ x y z i ⟧ᴾ = ×-distₗ-+ᴾ ⟦ x ⟧ᴾ ⟦ y ⟧ᴾ ⟦ z ⟧ᴾ i
  ⟦ ×-annihₗ x i ⟧ᴾ = ×-annihₗᴾ ⟦ x ⟧ᴾ i

  ⟦ trans p q i ⟧ᴾ = transᴾ (cong ⟦_⟧ᴾ p) (cong ⟦_⟧ᴾ q) i
  ⟦ embed x ⟧ᴾ = embedᴾ x
  ⟦ embed-zero x p i ⟧ᴾ = embed-zeroᴾ x p i
  ⟦ embed-suc x x' p i ⟧ᴾ = embed-sucᴾ x ⟦ x' ⟧ᴾ p i
\end{code}

Now we can extract the familiar \AgdaFunction{predᴱ} in
Section~\ref{sec:dsl-mul}, but also the proof that it is the predecessor.
\begin{code}
  predᴱ : Expr → Maybe Expr
  predᴱ x = pred ⟦ x ⟧ᴾ

  isPredᴱ : ∀ x → IsPred (predᴱ x) x
  isPredᴱ x = isPred ⟦ x ⟧ᴾ
\end{code}
\begin{code}[hide]
  pred-zero : predᴱ zero ≡ nothing
  pred-zero = refl

  pred-suc : ∀ x → predᴱ (suc x) ≡ just x
  pred-suc x = congS just (+-idₗ _)

  unpred-pred-pred : ∀ {x x'} → IsPred x' x → matchMaybe zero suc x' ≡ x
  unpred-pred-pred nothing = refl
  unpred-pred-pred (just x') = refl

  unpred-pred : ∀ x → matchMaybe zero suc (predᴱ x) ≡ x
  unpred-pred x = unpred-pred-pred (isPredᴱ x)
\end{code}

Using the new predecessor function, we can define a head normalisation function.
\begin{code}
  interpᴺ : Expr → NExpr
  interpᴺ x .pred = predᴱ x
\end{code}
Then we can prove that it is an inverse of \AgdaInductiveConstructor{embed} and
that \AgdaDatatype{Expr} and \AgdaRecord{NExpr} are isomorphic using
\AgdaFunction{isPredᴱ} and the \AgdaInductiveConstructor{embed-zero} and
\AgdaInductiveConstructor{embed-suc} constructors that we added to the language.
\begin{AgdaSuppressSpace}
\begin{code}
  interpᴺ-embed : ∀ x → interpᴺ (embed x) ≡ x
\end{code}
\begin{code}[hide]
  interpᴺ-embed x i .pred = x .pred
\end{code}
\begin{code}
  embed-interpᴺ : ∀ x → embed (interpᴺ x) ≡ x
\end{code}
\begin{code}[hide]
  embed-interpᴺ-pred : ∀ {x x'} → IsPred x' x → embed (interpᴺ x) ≡ x

  embed-interpᴺ x = embed-interpᴺ-pred (isPredᴱ x)
  embed-interpᴺ-pred nothing = embed-zero (interpᴺ zero) refl
  embed-interpᴺ-pred (just x') =
    embed-suc (interpᴺ (suc x')) x' (congS just (+-idₗ _))
\end{code}
\begin{code}
  ExprIsoNExpr : Iso Expr NExpr
\end{code}
\begin{code}[hide]
  ExprIsoNExpr = iso interpᴺ embed interpᴺ-embed embed-interpᴺ
\end{code}
\end{AgdaSuppressSpace}
\begin{code}[hide]
open E using (NExpr; Expr; pred; embed; trans; predᴱ; embed-interpᴺ; interpᴺ)

record ℕ∞ : Type where
  coinductive
  field pred : Maybe ℕ∞

open ℕ∞ public
\end{code}

Now we define a function that embeds conatural numbers into expressions using
the \AgdaInductiveConstructor{embedℕ∞} constructor.
\begin{code}[hide]
mutual
\end{code}
\begin{code}
  embedℕ∞ : ℕ∞ → Expr
  embedℕ∞ x = embed (embedℕ∞ᴺ x)

  embedℕ∞ᴺ : ℕ∞ → NExpr
  pred (embedℕ∞ᴺ x) = embedℕ∞-match (pred x)

  embedℕ∞-match : Maybe ℕ∞ → Maybe Expr
  embedℕ∞-match nothing    = nothing
  embedℕ∞-match (just x')  = just (embedℕ∞ x')
\end{code}
We can also interpret expressions into conatural numbers using
\AgdaFunction{predᴱ} as in Section~\ref{sec:dsl-mul}.
\begin{code}[hide]
mutual
\end{code}
\begin{code}
  interp : Expr → ℕ∞
  pred (interp x) = interp-match (predᴱ x)

  interp-match : Maybe Expr → Maybe ℕ∞
  interp-match nothing    = nothing
  interp-match (just x')  = just (interp x')
\end{code}

These turn out to be inverses. If we embed a conatural number in the language
and then interpret it back to a conatural number, we get back the conatural
number we started with.
\begin{code}[hide]
mutual
\end{code}
\begin{code}
  interp-embedℕ∞ : ∀ x → interp (embedℕ∞ x) ≡ x
  pred (interp-embedℕ∞ x i) =
    interp-embedℕ∞-match (pred x) i

  interp-embedℕ∞-match :
    ∀ x' → interp-match (embedℕ∞-match x') ≡ x'
  interp-embedℕ∞-match nothing    = refl
  interp-embedℕ∞-match (just x')  =
    cong just (interp-embedℕ∞ x')
\end{code}
If we interpret an expression from the language to conatural numbers and then we
embed it back to the language, then we get an expression equal to the one we
started with. Here is the point where we had to use the embedded transitivity
operation, because otherwise it would not be guarded.
\begin{code}[hide]
mutual
\end{code}
\begin{code}
  embedℕ∞-interp : ∀ x → embedℕ∞ (interp x) ≡ x
  embedℕ∞-interp x =
    trans
      (cong embed (embedℕ∞-interpᴺ x))
      (embed-interpᴺ x)

  embedℕ∞-interpᴺ :
    ∀ x → embedℕ∞ᴺ (interp x) ≡ interpᴺ x
  pred (embedℕ∞-interpᴺ x i) =
    embedℕ∞-interp-match (predᴱ x) i

  embedℕ∞-interp-match :
    ∀ x' → embedℕ∞-match (interp-match x') ≡ x'
  embedℕ∞-interp-match nothing    = refl
  embedℕ∞-interp-match (just x')  =
    cong just (embedℕ∞-interp x')
\end{code}
With the roundtrips, we get an isomorphism between \AgdaDatatype{Expr} and
\AgdaRecord{ℕ∞}.
\begin{code}
ExprIsoℕ∞ : Iso Expr ℕ∞
\end{code}
\begin{code}[hide]
ExprIsoℕ∞ = iso interp embedℕ∞ interp-embedℕ∞ embedℕ∞-interp
\end{code}

Now that we have an isomorphism, we get all the operations and equations in
\AgdaDatatype{Expr} for \AgdaRecord{ℕ∞} by the univalence
principle~\cite{hottbook}. To do that, we define a record with all the
operations and equations of a commutative semiring.
\begin{AgdaAlign}
\begin{code}
record ℕ∞Str (A : Type) (pred : A → Maybe A) : Type
\end{code}
\begin{code}[hide]
  where
  infixl 6 _+_
  infixl 7 _×_
\end{code}
For example, we add the following fields (the rest is omitted for brevity):
\begin{code}
  field
    isSetA   : isSet A
    _+_      : A → A → A
    +-assoc  : ∀ x y z → (x + y) + z ≡ x + (y + z)
    +-comm   : ∀ x y → x + y ≡ y + x
\end{code}
\begin{code}[hide]
    zero : A
    +-idₗ : ∀ x → zero + x ≡ x

    _×_ : A → A → A
    ×-assoc : ∀ x y z → (x × y) × z ≡ x × (y × z)
    ×-comm : ∀ x y → x × y ≡ y × x

    one : A
    ×-idₗ : ∀ x → one × x ≡ x

    ×-distₗ-+ : ∀ x y z → (x + y) × z ≡ x × z + y × z
    ×-annihₗ : ∀ x → zero × x ≡ zero

  suc : A → A
  suc = one +_
\end{code}
We can also add equations that relate the predecessor function with the semiring
operations.
\begin{code}
  field
    pred-zero    : pred zero ≡ nothing
    pred-suc     : ∀ x → pred (one + x) ≡ just x
    unpred-pred  :
      ∀ x → matchMaybe zero (one +_) (pred x) ≡ x
\end{code}
\end{AgdaAlign}

With the record defined, we can easily define it for \AgdaDatatype{Expr}.
\begin{code}
ℕ∞StrExpr : ℕ∞Str Expr predᴱ
\end{code}
\begin{code}[hide]
ℕ∞StrExpr =
  record
    { isSetA = E.isSetExpr;
      Expr;
      pred-zero = E.pred-zero;
      pred-suc = E.pred-suc;
      unpred-pred = E.unpred-pred }
\end{code}
Then, by transporting over the isomorphism using univalence, we get the same
record for \AgdaRecord{ℕ∞}, and thus getting every single operation and equation
in the record for conatural numbers at once. Note that it is possible to derive
this without univalence, but with more work for every single operation and
equation.
\begin{code}
ℕ∞Strℕ∞ : ℕ∞Str ℕ∞ pred
ℕ∞Strℕ∞ =
  transport
    (cong₂ ℕ∞Str (isoToPath ExprIsoℕ∞) predᴱ≡pred)
    ℕ∞StrExpr
\end{code}
\begin{code}[hide]
  where
  subst-Maybe :
    ∀ x' → subst Maybe (isoToPath ExprIsoℕ∞) x' ≡ interp-match x'
  subst-Maybe nothing = refl
  subst-Maybe (just x') = cong just (transportRefl _)
  predᴱ≡pred = ua→ λ x → toPathP (subst-Maybe _)

open ℕ∞Str ℕ∞Strℕ∞ renaming (isSetA to isSetℕ∞) public
\end{code}
Hence we have shown that conatural numbers form a commutative semiring.

\subsection{Exponentiation}
\input{latex/QuotientedExp}
